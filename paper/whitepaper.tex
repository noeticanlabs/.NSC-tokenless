\documentclass{article}

% --- Document Setup ---
\usepackage[margin=1in]{geometry}
\usepackage{amsmath}
\usepackage{graphicx}
\usepackage{hyperref}

% --- Title and Author ---
\title{Noetica Physics: A Resonant Field Theory (RFT) \\ \large White Paper Thesis v3.0}
\author{Mikey + GPT Research Group}
\date{2025-10-08}

% --- Document Begins ---
\begin{document}

\maketitle

% --- Abstract ---
\begin{abstract}
Noetica Physics (Resonant Field Theory, RFT) proposes that the fundamental architecture of the universe is \textbf{resonant rather than particulate}---a continuous field of coupled oscillations whose geometry determines observable matter, energy, and information. The theory reformulates physical interactions as \textit{phase relations} within a harmonic manifold. Where General Relativity describes curvature of spacetime and Quantum Mechanics quantizes energy states, RFT treats \textbf{phase coherence} as the unifying variable linking both domains. It directly addresses the long-standing paradox between relativity’s smooth, deterministic spacetime and quantum mechanics’ probabilistic, discontinuous nature. RFT posits that these conflicting pictures emerge from different limits of the same resonant substrate---curvature and coherence as dual expressions of one field. The central claim: \textit{all fields---gravitational, electromagnetic, quantum, or informational---are expressions of a universal phase field whose curvature encodes energy and whose coherence encodes order.}
\end{abstract}

% --- Appendix D: Formal Derivation ---
\appendix
% Shortened section title to prevent overfull hbox warning.
\section{Derivation of the Euler–Lagrange Equations (RFT)}

\paragraph{Conventions.} Spacetime is a smooth 3+1D manifold with metric \(g_{\mu\nu}\) (signature \((+,-,-,-))\), Levi–Civita tensor \(\varepsilon^{\mu\nu\rho\sigma}\) with \(\varepsilon^{0123}=+1\), and covariant derivative \(\nabla_\mu\). Natural units (\(\hbar=c=1\)). Indices are raised with \(g^{\mu\nu}\). Define the dual field strength \(\tilde{F}^{\mu\nu} := \frac{1}{2} \varepsilon^{\mu\nu\rho\sigma}F_{\rho\sigma}\).

\subsection{Fields and Lagrangian}
\paragraph{Fields.} Phase \(\theta(x) \in S^1\); gauge potential \(A_\mu(x)\); field strength \(F_{\mu\nu} := \partial_\mu A_\nu - \partial_\nu A_\mu\); covariant phase gradient \(D_\mu\theta := \partial_\mu\theta - A_\mu\). Stiffness tensor \(K^{\mu\nu}(x)\) is symmetric positive-definite.

\paragraph{Action.}
\begin{equation}
S[\theta, A] = \int d^4x \sqrt{|g|} \mathcal{L}_{\text{RFT}},
\end{equation}
where
\begin{equation}
\mathcal{L}_{\text{RFT}} = \frac{\kappa_1}{2} D_\mu\theta K^{\mu\nu} D_\nu\theta - \frac{\kappa_2}{4} F_{\mu\nu}F^{\mu\nu} - V(\theta) + J^\mu D_\mu\theta + \frac{\alpha}{8\pi}\theta F_{\mu\nu}\tilde{F}^{\mu\nu}.
\end{equation}

% Using \texorpdfstring to provide a text-only version for PDF bookmarks, fixing hyperref warnings.
\subsection{Variation with respect to \texorpdfstring{\(\theta\)}{theta}}
Using \(\delta D_\mu\theta = \partial_\mu\delta\theta\) and integrating by parts, the variation of the action is:
\begin{equation}
\delta S_\theta = \int d^4x \sqrt{|g|} \delta\theta \left\{ -\nabla_\nu(\kappa_1 K^{\nu\mu}D_\mu\theta + J^\nu) - \frac{\partial V}{\partial\theta} + \frac{\alpha}{8\pi}F_{\mu\nu}\tilde{F}^{\mu\nu} \right\}.
\end{equation}
This yields the \textbf{Phase Equation (Euler-Lagrange)}:
\begin{equation}
\nabla_\nu(\kappa_1 K^{\nu\mu}D_\mu\theta + J^\nu) + \frac{\partial V}{\partial\theta} = \frac{\alpha}{8\pi}F_{\mu\nu}\tilde{F}^{\mu\nu}.
\end{equation}

% Using \texorpdfstring to provide a text-only version for PDF bookmarks, fixing hyperref warnings.
\subsection{Variation with respect to \texorpdfstring{\(A_\mu\)}{A_mu}}
With \(\delta D_\mu\theta = -\delta A_\mu\) and \(\delta F_{\mu\nu} = \nabla_\mu\delta A_\nu - \nabla_\nu\delta A_\mu\), integrating by parts gives:
\begin{equation}
\delta S_A = \int d^4x \sqrt{|g|} \delta A_\mu \left\{ \kappa_2\nabla_\nu F^{\nu\mu} - \kappa_1 K^{\mu\nu}D_\nu\theta - J^\mu - \frac{\alpha}{2\pi}(\partial_\nu\theta) \tilde{F}^{\nu\mu} \right\}.
\end{equation}
This yields the \textbf{Gauge Equation (Euler-Lagrange)}:
\begin{equation}
\nabla_\nu F^{\nu\mu} = \frac{1}{\kappa_2} \left( \kappa_1 K^{\mu\nu}D_\nu\theta + J^\mu + \frac{\alpha}{2\pi} (\partial_\nu\theta) \tilde{F}^{\nu\mu} \right).
\end{equation}


\end{document}
